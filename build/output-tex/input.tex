% Options for packages loaded elsewhere
\PassOptionsToPackage{unicode,linktoc=all,hidelinks}{hyperref}
\PassOptionsToPackage{hyphens}{url}
\PassOptionsToPackage{dvipsnames,svgnames,x11names}{xcolor}
%
\documentclass[
  12pt,
  paper=a4,
  twoside,
  titlepage=true,
  openright,
  abstract=on,
  toc=listofnumbered,
  numbers=noenddot,
  chapterprefix=true,
  headings=optiontohead,
  svgnames,
  dvipsnames]{scrreprt}
\usepackage{amsmath,amssymb}
\usepackage{lmodern}
\usepackage{iftex}
\ifPDFTeX
  \usepackage[T1]{fontenc}
  \usepackage[utf8]{inputenc}
  \usepackage{textcomp} % provide euro and other symbols
\else % if luatex or xetex
  \usepackage{unicode-math}
  \defaultfontfeatures{Scale=MatchLowercase}
  \defaultfontfeatures[\rmfamily]{Ligatures=TeX,Scale=1}
  \setmainfont[]{Latin Modern Roman}
  \setsansfont[]{Latin Modern Sans}
  \setmonofont[]{Latin Modern Mono}
  \setmathfont[]{Latin Modern Math}
\fi
% Use upquote if available, for straight quotes in verbatim environments
\IfFileExists{upquote.sty}{\usepackage{upquote}}{}
\IfFileExists{microtype.sty}{% use microtype if available
  \usepackage[]{microtype}
  \UseMicrotypeSet[protrusion]{basicmath} % disable protrusion for tt fonts
}{}
\makeatletter
\@ifundefined{KOMAClassName}{% if non-KOMA class
  \IfFileExists{parskip.sty}{%
    \usepackage{parskip}
  }{% else
    \setlength{\parindent}{0pt}
    \setlength{\parskip}{6pt plus 2pt minus 1pt}}
}{% if KOMA class
  \KOMAoptions{parskip=half}}
\makeatother
\usepackage{xcolor}
\IfFileExists{xurl.sty}{\usepackage{xurl}}{} % add URL line breaks if available
\IfFileExists{bookmark.sty}{\usepackage{bookmark}}{\usepackage{hyperref}}
\hypersetup{
  pdftitle={Technical Documents},
  pdfauthor={Gabriel Nützi; The Community},
  pdflang={en-GB},
  colorlinks=true,
  linkcolor={DarkBlue},
  filecolor={DarkBlue},
  citecolor={DarkBlue},
  urlcolor={MediumBlue},
  pdfcreator={LaTeX via pandoc}}
\urlstyle{same} % disable monospaced font for URLs
\usepackage{color}
\usepackage{fancyvrb}
\newcommand{\VerbBar}{|}
\newcommand{\VERB}{\Verb[commandchars=\\\{\}]}
\DefineVerbatimEnvironment{Highlighting}{Verbatim}{commandchars=\\\{\}}
% Add ',fontsize=\small' for more characters per line
\newenvironment{Shaded}{}{}
\newcommand{\AlertTok}[1]{\textcolor[rgb]{1.00,0.00,0.00}{\textbf{#1}}}
\newcommand{\AnnotationTok}[1]{\textcolor[rgb]{0.38,0.63,0.69}{\textbf{\textit{#1}}}}
\newcommand{\AttributeTok}[1]{\textcolor[rgb]{0.49,0.56,0.16}{#1}}
\newcommand{\BaseNTok}[1]{\textcolor[rgb]{0.25,0.63,0.44}{#1}}
\newcommand{\BuiltInTok}[1]{#1}
\newcommand{\CharTok}[1]{\textcolor[rgb]{0.25,0.44,0.63}{#1}}
\newcommand{\CommentTok}[1]{\textcolor[rgb]{0.38,0.63,0.69}{\textit{#1}}}
\newcommand{\CommentVarTok}[1]{\textcolor[rgb]{0.38,0.63,0.69}{\textbf{\textit{#1}}}}
\newcommand{\ConstantTok}[1]{\textcolor[rgb]{0.53,0.00,0.00}{#1}}
\newcommand{\ControlFlowTok}[1]{\textcolor[rgb]{0.00,0.44,0.13}{\textbf{#1}}}
\newcommand{\DataTypeTok}[1]{\textcolor[rgb]{0.56,0.13,0.00}{#1}}
\newcommand{\DecValTok}[1]{\textcolor[rgb]{0.25,0.63,0.44}{#1}}
\newcommand{\DocumentationTok}[1]{\textcolor[rgb]{0.73,0.13,0.13}{\textit{#1}}}
\newcommand{\ErrorTok}[1]{\textcolor[rgb]{1.00,0.00,0.00}{\textbf{#1}}}
\newcommand{\ExtensionTok}[1]{#1}
\newcommand{\FloatTok}[1]{\textcolor[rgb]{0.25,0.63,0.44}{#1}}
\newcommand{\FunctionTok}[1]{\textcolor[rgb]{0.02,0.16,0.49}{#1}}
\newcommand{\ImportTok}[1]{#1}
\newcommand{\InformationTok}[1]{\textcolor[rgb]{0.38,0.63,0.69}{\textbf{\textit{#1}}}}
\newcommand{\KeywordTok}[1]{\textcolor[rgb]{0.00,0.44,0.13}{\textbf{#1}}}
\newcommand{\NormalTok}[1]{#1}
\newcommand{\OperatorTok}[1]{\textcolor[rgb]{0.40,0.40,0.40}{#1}}
\newcommand{\OtherTok}[1]{\textcolor[rgb]{0.00,0.44,0.13}{#1}}
\newcommand{\PreprocessorTok}[1]{\textcolor[rgb]{0.74,0.48,0.00}{#1}}
\newcommand{\RegionMarkerTok}[1]{#1}
\newcommand{\SpecialCharTok}[1]{\textcolor[rgb]{0.25,0.44,0.63}{#1}}
\newcommand{\SpecialStringTok}[1]{\textcolor[rgb]{0.73,0.40,0.53}{#1}}
\newcommand{\StringTok}[1]{\textcolor[rgb]{0.25,0.44,0.63}{#1}}
\newcommand{\VariableTok}[1]{\textcolor[rgb]{0.10,0.09,0.49}{#1}}
\newcommand{\VerbatimStringTok}[1]{\textcolor[rgb]{0.25,0.44,0.63}{#1}}
\newcommand{\WarningTok}[1]{\textcolor[rgb]{0.38,0.63,0.69}{\textbf{\textit{#1}}}}
\usepackage{longtable,booktabs,array}
\usepackage{calc} % for calculating minipage widths
% Correct order of tables after \paragraph or \subparagraph
\usepackage{etoolbox}
\makeatletter
\patchcmd\longtable{\par}{\if@noskipsec\mbox{}\fi\par}{}{}
\makeatother
% Allow footnotes in longtable head/foot
\IfFileExists{footnotehyper.sty}{\usepackage{footnotehyper}}{\usepackage{footnote}}
\makesavenoteenv{longtable}
\setlength{\emergencystretch}{3em} % prevent overfull lines
\providecommand{\tightlist}{%
  \setlength{\itemsep}{0pt}\setlength{\parskip}{0pt}}
\setcounter{secnumdepth}{3}
\newlength{\cslhangindent}
\setlength{\cslhangindent}{1.5em}
\newlength{\csllabelwidth}
\setlength{\csllabelwidth}{3em}
\newlength{\cslentryspacingunit} % times entry-spacing
\setlength{\cslentryspacingunit}{\parskip}
\newenvironment{CSLReferences}[2] % #1 hanging-ident, #2 entry spacing
 {% don't indent paragraphs
  \setlength{\parindent}{0pt}
  % turn on hanging indent if param 1 is 1
  \ifodd #1
  \let\oldpar\par
  \def\par{\hangindent=\cslhangindent\oldpar}
  \fi
  % set entry spacing
  \setlength{\parskip}{#2\cslentryspacingunit}
 }%
 {}
\usepackage{calc}
\newcommand{\CSLBlock}[1]{#1\hfill\break}
\newcommand{\CSLLeftMargin}[1]{\parbox[t]{\csllabelwidth}{#1}}
\newcommand{\CSLRightInline}[1]{\parbox[t]{\linewidth - \csllabelwidth}{#1}\break}
\newcommand{\CSLIndent}[1]{\hspace{\cslhangindent}#1}
\ifLuaTeX
\usepackage[bidi=basic]{babel}
\else
\usepackage[bidi=default]{babel}
\fi
\babelprovide[main,import]{british}
% get rid of language-specific shorthands (see #6817):
\let\LanguageShortHands\languageshorthands
\def\languageshorthands#1{}
% Set include paths
\makeatletter
\providecommand*{\input@path}{}
\edef\input@path{{tools/convert/includes/}{./}\input@path}% prepend
\makeatother

% Calculations =======================================
\usepackage{etoolbox}
\usepackage{calc}
% ====================================================

%% Geometry ==========================================
\newlength{\innermargin}
\setlength{\innermargin}{2.1cm}
\usepackage[nomarginpar,
            textwidth=16.5cm,
            textheight=240mm,
            bottom=2.5cm,
            headheight=15pt,
            headsep=30pt,
            footskip=30pt,
            bindingoffset=7mm,
            inner=\innermargin,
            twoside
            ]{geometry}
% ====================================================

%% Grafics ===========================================
\usepackage{graphicx}
\usepackage{tikz}
\usepackage[export]{adjustbox}
\usepackage[final]{pdfpages}
\usepackage{wrapfig}
\usepackage[section]{placeins} % float barrier at
                               % each section
\usepackage{svg}
\svgsetup{extractpath=files/generated/svg-extract, inkscapepath=files/generated/svg-inkscape}
% ====================================================


%% Caption ===========================================
\usepackage[%
    format=plain,%
    justification=justified,%
    margin=3mm,%
    font=normal,%
    labelfont=bf%
]{caption}

\DeclareCaptionStyle{mycaption}{
    format=plain,%
    justification=justified,%
    margin=12pt,%
    font=normal,%
    labelfont=bf,%
    %aboveskip=10pt,
    belowskip=-5pt%
}

\usepackage[]{subcaption}
\captionsetup[figure]{style=mycaption}
% ====================================================

%% Header/Footer =====================================
\usepackage[automark,headsepline]{scrlayer-scrpage}
\clearpairofpagestyles
\setkomafont{pagehead}{\normalfont\rmfamily\bfseries}
%\setkomafont{pagenumber}{\normalfont\rmfamily}

\automark[chapter]{chapter}
\renewcommand{\chaptermarkformat}{}
\renewcommand{\sectionmarkformat}{\thesection\autodot\enskip}
\cfoot[\pagemark]{\pagemark}
\lehead{Chapter \thechapter}
\rohead{\headmark}
\rehead{}
\lohead{}

% Footnotes
\usepackage[perpage,para]{footmisc}
% ====================================================

%% Tables ============================================
\usepackage{array}
\usepackage{longtable}
\usepackage{booktabs}
\usepackage{collcell}
\usepackage{makecell}
\usepackage{footnotehyper}
\usepackage{tablefootnote}
\usepackage{enumitem}
\AtBeginEnvironment{longtable}{%
\setlist[itemize]{nosep=0pt,
                 leftmargin=*,
                 label=\textbullet,
                 after=\end{minipage},
                 before=\begin{minipage}[t]{\linewidth}
}
}

% Define default table spacing...
\usepackage{cellspace}
\setlength{\cellspacetoplimit}{0.2\baselineskip}
\setlength{\cellspacebottomlimit}{0.2\baselineskip}
% ====================================================

%% Quotes =============================================
\usepackage[font=itshape,vskip=5pt]{quoting}

%% Todo Notes =========================================
% \usepackage[textsize=tiny,disable]{todonotes}
% \newcommand{\disscomment}[1]{%
%     \todo[color=black!40]{#1}%
% }%

%% Glossary ==========================================
% \usepackage[hyperfirst=true]{glossaries}
% \glstoctrue
% \renewcommand*{\glstextformat}[1]{\textnormal{#1}}
% \newcommand{\glsp}{\protect\gls}
% \loadglsentries{chapter/GlossaryEntries}
%\makeglossaries
% ====================================================

%% Own styles! ========================================
\ProvidesPackage{GeneralMacros}[general style definitions]
\RequirePackage{graphicx}
\RequirePackage{xparse}
\RequirePackage{svg}

% Image with caption
\NewDocumentCommand\imageWithCaption{mmmo}{
\begin{figure}[h]
    \centering
    \adjustbox{center}{\adjustbox{cfbox=white!90!black 2pt 0pt}{\includegraphics[#3]{#1}}}
    \caption{#2}
    \IfValueT{#4}{\label{#4}}
\end{figure}
}

% SVG with caption
\NewDocumentCommand\svgWithCaption{mmmo}{
\begin{figure}[h]
    \centering
    \adjustbox{center}{\adjustbox{cfbox=white!90!black 2pt 0pt}{\includesvg[inkscapearea=page,#3]{#1}}}
    \caption{#2}
    \IfValueT{#4}{\label{#4}}
\end{figure}
}

\newfontfamily\DejaSans{DejaVu Sans}
\NewDocumentCommand\emojiFont{m}{
{\DejaSans #1}%
}

\NewDocumentCommand{\includePDF}{m O{1} m O{}}
{%
    \foreach \x in {#2,...,#3} {%
        \begin{center}%
        \makebox[\textwidth]{\includegraphics[page=\x, #4]{#1}}%
        \end{center}%
        \clearpage%
    }
}

\ProvidesPackage{MathMacros}[math style definitions]

\RequirePackage{tikz}
\usepackage[customcolors,markings]{hf-tikz}
\usetikzlibrary{calc,tikzmark}

% Left Subscripts
\RequirePackage{leftidx}

% Math
\RequirePackage{mathdots}

% Markdown converted definitions
\newcommand{\vvec}[1]{\mathbf{#1}}
\newcommand{\mat}[1]{\mathbf{#1}}
\newcommand{\cs}[1]{\mathrm{#1}}
\newcommand{\rp}[1]{{_\cs{#1}}}
\newcommand{\csT}[2]{\mat{A}_{\mathrm{#1#2}}}
\newcommand{\genT}[2]{\mat{T}_{\mathrm{#1#2}}}
\newcommand{\affcsT}[2]{\mat{H}_{\mathrm{#1#2}}}
\newcommand{\affcsTdx}[2]{\mat{H}_{\mathrm{#1#2}}^{\transp}}
\newcommand{\basevec}[2]{\vvec{e}_{\mathrm{#1}}^{\cs{#2}}}
\newcommand{\homArr}[1]{\left[\begin{array}{c}#1 \\ 1 \end{array}\right]}
\newcommand{\homArrTr}[1]{\left[\begin{array}{c}#1 && 1 \end{array}\right]}
\newcommand{\homMat}[2]{\left[\begin{array}{cc}#1 & #2 \\ \vvec{0} & 1 \end{array}\right]}
\newcommand{\arr}[2]{\left[\begin{array}{#1}#2\end{array}\right]}
\newcommand{\transp}{\top}
\newcommand{\rot}[2]{\mat{R}_\mathrm{#1#2}}
\newcommand{\norm}[1]{\|#1\|}
\newcommand{\mdet}[1]{\det(#1)}
\newcommand{\set}[1]{\mathcal{#1}}
\newcommand{\prox}[1]{\mathbf{prox}_{\set{C}}}
\DeclareMathOperator*{\argmin}{argmin}
\newcommand{\ncone}[1]{\mathcal{N}_{\set{#1}}}
\newcommand{\indf}[1]{I_{\set{#1}}}

% ====================================================

%% Custom Part Heading ===============================
\addtokomafont{part}{\Huge\selectfont\rmfamily\bfseries}
\addtokomafont{partprefix}{\Large\rmfamily\bfseries}
%\newcommand*\partcolor{\color{blue!50}}% Part is coloured blue
\renewcommand*\partheadstartvskip{\vspace*{.2\textheight}}
\renewcommand*\partheadmidvskip{%
    \par
    \vspace*{30pt}%
}
\makeatletter
\renewcommand*\partheadendvskip{%
    \vspace{\baselineskip}%
    \partquotenote%
    \vfil\newpage%
    \if@twoside%
        \if@openright%
        \null%
        \thispagestyle{plain}%
        \vspace*{\fill}%
        \partnote%
        \vspace*{\fill}%
        \fi%
    \else%
         \thispagestyle{plain}%
         \vspace*{\fill}%
         \partnote%
         \vspace*{\fill}%
         \newpage%
    \fi%
    \if@tempswa%
    \twocolumn%
    \fi%
}
\makeatother

\newcommand\partnote{}
\newcommand\partquotenote{}
\renewcommand\partformat{\hfill\color{lightgray}\partname~\fontsize{60}{60}\selectfont\thepart\if@altsecnumformat.\fi}
%=====================================================

%% Custom Chapter Heading ============================
\renewcommand*\chapterheadstartvskip{\vspace*{.02\textheight}}
\renewcommand*\chapterheadendvskip{%
    \noindent{\setlength{\parskip}{0pt}\hrulefill\par}%
    \vspace*{30pt}%
}
\renewcommand*{\chapterformat}{%
    \parbox{\textwidth}{\hfill\chapappifchapterprefix{\ }\fontsize{85}{70}\selectfont\thechapter\autodot\enskip}%
    \vspace{4ex}%
    }
\addtokomafont{disposition}{\normalfont\bfseries}
\addtokomafont{chapterprefix}{\Large\bfseries\color{lightgray}}
%=====================================================

%% Part Quotes =======================================
\setkomafont{dictumtext}{\itshape\small}
\setkomafont{dictumauthor}{\normalfont}
\renewcommand*\dictumwidth{0.9\linewidth}
\renewcommand*\raggeddictum{\centering}
\renewcommand*\dictumauthorformat[1]{--- #1}
\renewcommand*\dictumrule{
    %\vskip-1ex\hrulefill\par}
}
%=====================================================

%% Heading fonts =====================================
\setkomafont{chapter}{\huge\rmfamily}
\addtokomafont{section}{\rmfamily}
\addtokomafont{subsection}{\rmfamily}
\addtokomafont{subsubsection}{\rmfamily}
%=====================================================

%% Heading spacing ===================================
% \RedeclareSectionCommand[
%   runin=false,
%   afterindent=false,
%   beforeskip=1.0\baselineskip,
%   afterskip=0.3\baselineskip]{section}
% \RedeclareSectionCommand[
%   runin=false,
%   afterindent=false,
%   beforeskip=0.5\baselineskip,
%   afterskip=0.0\baselineskip]{subsection}
% \RedeclareSectionCommand[
%   runin=false,
%   afterindent=false,
%   beforeskip=0.5\baselineskip,
%   afterskip=0.0\baselineskip]{subsubsection}
%=====================================================

%% Table of Content ==================================
\usepackage{titling}
% \usepackage{titletoc}
% \newcommand{\setupTOC}[2]{%
%     \titlecontents{#1}
%     [8em] % ie, 1.5em (chapter) + 2.3em
%     {\rightskip=10mm plus 1fil\hyphenpenalty=10000\normalsize\bfseries\protect\addvspace{15pt}\contentsmargin{3em}}
%     {\contentslabel[{\color{gray}#2\enspace\thecontentslabel}]{8em}}
%     {\hspace*{-8em}}
%     {\hfill\contentspage}
% }
% \setupTOC{chapter}{\chaptername}
% =====================================================

% Layout Placing Settings ==============================
% \setcounter{topnumber}{2}
% \setcounter{bottomnumber}{2}
% \setcounter{totalnumber}{3}     % 2 may work better
% \setcounter{dbltopnumber}{2}    % for 2-column pages
\renewcommand{\topfraction}{0.85}
\renewcommand{\bottomfraction}{0.8}
\renewcommand{\textfraction}{0.07}
\renewcommand{\floatpagefraction}{0.8}
%\renewcommand{\dbltopfraction}{.66}
%\renewcommand{\dblfloatpagefraction}{.8}

% Allow display break
%\allowdisplaybreaks[3]
%=======================================================

% Tocless Command ======================================
\newcommand{\nocontentsline}[3]{}
\newcommand{\tocless}[2]{\bgroup\let\addcontentsline=\nocontentsline#1{#2}\egroup}

% Clever Refs ==========================================
\usepackage[capitalise]{cleveref} % load it after thmtools (there is some problem)
\crefname{appsec}{appendix}{appendices}
%=======================================================

% General Settings
\KOMAoptions{cleardoublepage=empty}
\raggedbottom
\ifLuaTeX
  \usepackage{selnolig}  % disable illegal ligatures
\fi

\title{Technical Documents}
\usepackage{etoolbox}
\makeatletter
\providecommand{\subtitle}[1]{% add subtitle to \maketitle
  \apptocmd{\@title}{\par {\large #1 \par}}{}{}
}
\makeatother
\subtitle{Demonstrating the Power of Markdown with Pandoc}
\author{Gabriel Nützi \and The Community}
\date{2. December 2020}
\begin{document}
\begin{titlepage}
\makeatletter
\begin{center}
\includesvg[inkscapearea=page,width=0.5\textwidth]{files/Logo.svg}
\end{center}
\vspace{1cm}
\begin{center}
\Huge \textbf{\@title}
\end{center}
\begin{center}
\LARGE Demonstrating the Power of Markdown with Pandoc
\end{center}
\vspace{1cm}
\begin{center}
\textbf{Gabriel Nützi} \\ \& \\ \textbf{The Community}
\end{center}
\begin{center}
\@date
\end{center}
\begin{center}
\vfill
Zürich, Switzerland
\vspace{1cm}
\end{center}
\makeatother
\end{titlepage}
\begin{abstract}
This is a setup demonstrating the power and use of markdown for
technical documents by using a fully automated conversion sequence with
\href{https://gradle.org}{\texttt{gradle}} and of course
\href{https://pandoc.org}{\texttt{pandoc}}.
\end{abstract}

{
\hypersetup{linkcolor=Black}
\setcounter{tocdepth}{2}
\tableofcontents
}
\hypertarget{intro}{%
\chapter{Intro}\label{intro}}

Read the
\href{https://github.com/gabyx/TechnicalMarkdown/blob/master/Readme.md}{Readme.md}
for futher information.

\hypertarget{samples}{%
\chapter{Samples}\label{samples}}

\hypertarget{konvexe-probleme}{%
\section{Konvexe Probleme}\label{konvexe-probleme}}

In diesem Abschnitt geht es darum ein besseres Verständnis zu geben über
Algorithmen und Konzepte welche zum Beispiel bei der
\(2\)d-/\(3\)d-Kollisionsdetektion oder in Optimierungs-Algorithmen
genutzt werden. Die Erklärungen sind eine stärkere und vereinfachte
Zusammenfassung aus \protect\hyperlink{ref-nuetzig_thesis_2016}{{[}1,
Ch. 6{]}}. Dieses Kapitel möchte nicht mathematisch abschliessend sein,
sondern lediglich (nach der Ansicht des Autors) ein paar wichtige
Grundkonzepte vermitteln, welche einen wunderbaren Einstieg in dieses
Thema geben. Die erwähnten Konzepte sind in der generellsten Form der
konvexen Analysis zu finden und können sehr wohl als die wichtigsten
Standbeine verinnerlicht werden. Auf Beweise wird bewusst verzichtet. Es
wird versucht die mathematischen Definitionen anschaulich zu erklären.

Das Vorhandensein eines \emph{konvexen} Optimierungsproblems oder auch
einer \emph{konvexen Menge} beim Lösen eines Problems in \(3\)D, oder
auch mehr Dimensionen, erlaubt es, auf eine Fülle von mathematisch sehr
etablierten Definitionen und Konzepte aus der \emph{konvexen Analysis}
zurückzugreifen. Diese schon fast 50-jährige Theorie ist sehr fundiert
und ein abgeschlossenes Untergebiet in der Mathematik.

Die \emph{konvexe Analysis} ist ein Grenzgebiet von \emph{Geometrie},
\emph{Analysis} und \emph{Funktionalanalysis}, das sich mit den
Eigenschaften \textbf{konvexer Mengen} und \textbf{konvexer Funktionen}
befaßt und Anwendungen sowohl in der reinen Mathematik besitzt (von
Existenzsätzen in der Theorie der Differential- und Integralgleichungen
bis zum Gitterpunktsatz von Minkowski in der Zahlentheorie) als auch in
Bereichen wie der mathematischen Ökonomie und den
Ingenieurswissenschaften, wo man es oft mit Optimierungs- und
Gleichgewichtsproblemen zu tun hat. Als einschlägige Referenz auf diesem
Gebiet sei hier mal das Standardwerk
\protect\hyperlink{ref-rockafellar_convex_2015}{{[}2{]}} gegeben.

\hypertarget{konvexe-menge}{%
\subsection{Konvexe Menge}\label{konvexe-menge}}

Eine Menge \(\set{C} \subseteq V\), also ein Teilmenge eines Vektorraums
\(V\), wird \textbf{konvex} genannt, falls und nur falls
\begin{align} \lambda \vvec{a} + (1-\lambda) \vvec{b} \in \set{C} \quad \forall \vvec{a},\vvec{b} \in \set{C}, \ \lambda \in [0,1] \ . \end{align}
gilt.

Das heisst alle Punkte auf einer geraden Linie zwischen zwei beliebigen
Punkten aus der Menge \(\set{C}\) müssen \textbf{auch} in dieser Menge
liegen damit es \textbf{konvex} ist. Stellt man sich eine Banane vor
oder eine Oberfläche eines \(3\)d-Würfels, dann erfüllen diese Mengen
das Kriterium nicht. Ein ausgefülltes \(2\)d-Rechteck , ein gefüllter
\(3\)d-Würfel oder eine gefüllte Kugel jedoch schon. Eine
\textbf{konvexe} Menge ist eine Teilmenge eines Vektorraums. Ein
Beispiel ist der euklidische Raum \(\mathbb{E}^3\) in \(3\)D.

Man interessiert sich nun zum Beispiel für die Projektion eines Punktes
\(\vvec{x}\) auf ein konvexe Menge \(\set{C}\). Die Projektion sollte so
sein, dass der Abstand zwischen dem Punkt \(\vvec{x}\) und dem
projezierten Punkt \(\vvec{y} \in \set{C}\) \textbf{minimal} ist.

Der Abstand wird über eine Metrik \(d(\vvec{x},\vvec{y}) \geq 0\)
definiert - die \emph{Distanzfunktion}. Der euklidische Vektorraum
\(\mathbb{E}^3\) ist ein Vektorraum mit einer Metrik und ist daher ein
\emph{metrischer Raum}. Das Standard-Skalarprodukt
\(\vvec{x}^\transp \vvec{y}\) im euklidischen Raum \(\mathbb{E}^3\)
induziert direkt die Standardnorm
\(\norm{\vvec{x}}_2 := \sqrt{\vvec{x}^\transp \vvec{x}} \geq 0\). Diese
wiederum induziert direkt die Metrik
\begin{align} d(\vvec{x},\vvec{y}) := \norm{\vvec{x} - \vvec{y}}_2. \end{align}

Wir können somit über die Metrik \(d(\vvec{x},\vvec{y})\) die Länge
zwischen zwei Vektoren \(\vvec{x},\vvec{y} \in \mathbb{E}^3\) berechnen.

\hypertarget{proximaler-punkt}{%
\subsection{Proximaler Punkt}\label{proximaler-punkt}}

Mit der Metrik, also unserem Lineal zum Messen von Distanzen, lässt sich
nun die Projektion \(\prox{C}(\vvec{p})\) eines Punktes \(\vvec{p}\) mit
minimaler Distanz auf eine konvexe Menge \(\set{C}\) relativ leicht
definieren zu
\begin{align} \prox{C}(\vvec{p}) :=  \underset{\vvec{x} \ \in \ \set{C}}{\argmin} \norm{\vvec{x} - \vvec{p}}_2. \end{align}

Das heisst \(\prox{C}(\vvec{p})\) minimiert den Punkt \(\vvec{x}\) in
der Menge \(\set{C}\) so, dass sein Abstand zu \(\vvec{p}\) minimal ist.

\hypertarget{normalkegel}{%
\subsection{Normalkegel}\label{normalkegel}}

Eines der \textbf{wichtigsten} Konzept der konvexen Analysis ist die des
\textbf{Normalkegels}. Wie der Name schon sagt, handelt es sich um einen
Kegel welcher durch Normalenvektoren auf der Oberfläche einer konvexen
Menge aufgespannt wird. Wir geben hier direkt die Definition und sehen
im Anschluss wie sich dieser Kegel visualisiert:

Ein Normalkegel \(\ncone{C}\) auf ein konvexes Set \(\set{C}\) im Punkt
\(\vvec{x} \in \set{C}\) ist definert als
\begin{align}     \ncone{C}(\vvec{x}) := \left\{ \vvec{y} \ | \ \vvec{y}^\transp(\vvec{x}^* - \vvec{x}) \leq 0, \quad \forall \vvec{x}^* \in \set{C} \right\} \end{align}

Das ist nun ein wenig kryptisch, heisst jedoch nichts anderes als
folgendes: Der Normalkegel \(\ncone{C}(\vvec{x})\) besteht aus allen
Vektoren (das wäre \(\vvec{y}\)) ausgehend von \(\vvec{x}\) welche mit
\textbf{allen} Vektoren welche vom Punkt \(\vvec{x}\) in die Menge
\(\set{C}\) zeigen (das wäre \(\vvec{x}^* - \vvec{x}\)), einen
\textbf{stumpfen} Winkel bilden (das wäre das Skalatprodukt mit
\(\leq 0\)). Der Ursprung der Menge \(\ncone{C}(\vvec{x})\) ist im Punkt
\(\vvec{x}\).

Die Abbildung \labelcref{fig:normalcone} visualisiert für eine konvexe
Menge \(\set{C}\) die verschiedenen Normalkegel.

\svgWithCaption{files/NormalKegel.svg}{Normalkegel an die Punkte
\(\vvec{x}\), \(\vvec{y}\) und \(\vvec{z}\). Der Normalkegel an einen
innerhalb der Menge \(\set{C}\) liegenden Punkt \(\vvec{z}\) degeneriert
zum \(\vvec{0}\)-Vektor. Der Vektor \(\vvec{v}\) ist in der Menge des
Normalkegels an \(\vvec{x}\).}{}[fig:normalcone]

\hypertarget{zusammenhang-von-normalkegel-und-proximaler-punkt}{%
\subsection{Zusammenhang von Normalkegel und Proximaler
Punkt}\label{zusammenhang-von-normalkegel-und-proximaler-punkt}}

Man fragt sich natürlich nun: \emph{Was bringen uns diese mathematische
Definitionen?}

Es stellt sich heraus, dass es einen Zusammenhang gibt zwischen
\(\prox{C}\) und \(\ncone{C}\) welcher extremst nützlich ist und heute
im Feld der konvexen Optimierung, beim Machine-Learning, in der
Starrkörper-Mechanik (Starrkörper-Simulationen und Physics-Engines in
Games) oder auch in der Kollisionsdetektion (GJK Algorithmus) durch
projektive Iterationen direkte Anwendung findet.

Der Zusammenhang ist wie folgt:
\begin{align} \vvec{y} \in \ncone{C}(\vvec{x}) \quad \Leftrightarrow \quad \vvec{x} = \prox{C}(\vvec{x} + \vvec{y}) \label{eq:prox-to-ncone} \end{align}

Das heisst, eine Normalkegel-\emph{Inklusion} (die Relation
\(\vvec{a} \in \set{B}\) wird \emph{Mengen-Inklusion} genannt) ist
direkt an eine \textbf{implizite} \emph{projektive} Gleichung gekoppelt.

Damit lässt sich nun ein interessanter wichtiget Fakt ableiten. Aus der
Visualisierung \labelcref{fig:normalcone} entnehmen wir, dass
\(\vvec{p}-\vvec{x}\) in der Menge \(\ncone{C}(\vvec{x})\) liegt, also
lässt sich schreiben
\begin{align} \vvec{p}-\vvec{x} \in \ncone{C}(\vvec{x}). \end{align}

Dies lässt sich mit obiger Beziehung direkt zu
\begin{align} \vvec{x} &= \prox{C}(\vvec{x} + \vvec{p} - \vvec{x}) \\ &= \prox{C}(\vvec{p}). \end{align}
umschreiben. Aus dem erkennen wir, dass der Ursprung des Normalkegels,
worin ein \textbf{beliebiger} Punkt \(\vvec{p}\) liegt, direkt der
\textbf{proximale} Punkt ist zu \(\vvec{p}\).

Müssten wir nun eine Projektionsfunktion auf ein \(2\)d-Dreieck
herleiten, würden wir folgendes Bild malen:

\svgWithCaption{files/NormalKegelDreieck.svg}{Normalkegel an die Punkte
\(\vvec{a}\), \(\vvec{b}\) und \(\vvec{c}\) eines
Dreiecks.}{}[fig:normalconetri]

Das heisst es gibt genau 3 nicht triviale Normalkegel und 3 einfachere
Normalkegel (bestehend lediglich aus den Normalen auf die
Seitenflächen). Eine Projektionsfunktion auf ein Dreieck muss diese 6
Bereiche beachten und ist so auch optimal und richtig implementiert.

\hypertarget{zusammenhang-von-normalkegel-und-konvexer-optimierung}{%
\subsection{Zusammenhang von Normalkegel und Konvexer
Optimierung}\label{zusammenhang-von-normalkegel-und-konvexer-optimierung}}

Um hier mathematisch nicht in einen Exzess zu geraten, wird hier nur
eine abgespeckte Erklärung gegeben. Für mehr Informationen sei auf
\protect\hyperlink{ref-nuetzig_thesis_2016}{{[}1, Ch. 6{]}} verwiesen
und die darin enthaltenen Referenzen.

Betrachte man folgendes allgemeine restriktierte \textbf{konvexe}
Optimierungsproblem:
\begin{align} \vvec{x}^* = \underset{\vvec{x} \ \in \ \set{C}}{\argmin} f(\vvec{x}), \label{eq:convexproblem} \end{align}
wobei die Funktion \(f(\vvec{x}) \in \mathbb{R}\) \textbf{strikt konvex}
und \textbf{differenzierbar} (man stelle sich den oberen Teil eines
Weinglases vor, wobei \(\vvec{x} \in \mathbb{R}^2\)) ist und die
minimierenden Punkte \(\vvec{x}\) auf eine \textbf{konvexe} Menge
\(\set{C}\) restriktiert sind. Der minimierende Punkt ist hier mit
\(\vvec{x}^*\) bezeichnet. Es gibt nur \textbf{einen} solchen globalen
minimierenden Punkt

Dann kann man das Problem in ein freies \textbf{konvexes} Programm
umschreiben indem man die Einschränkung \(\vvec{x} \in \set{C}\) mit
einer Bestrafungsfunktion \(\indf{C}(\vvec{x})\) ersetzt
\begin{align} \vvec{x}^* = \underset{\vvec{x}}{\argmin} f(\vvec{x}) + \indf{C}(\vvec{x}). \end{align}

Die Bestrafungsfunktion \(\indf{C}(\vvec{x})\) liefert \(0\) falls
\(\vvec{x} \in \set{C}\) und sonst \(+\infty\). Diese Funktion wird
\textbf{Indikatorfunktion} genannt.

Die Frage ist nun wie kriegen wir eine Bedingung an den optimalen
(minimierenden) Punkt \(\vvec{x}^*\). Das geht ziemlich analog zu der
Bedindung für Minima/Maxima einer differenzierbaren Funktionen \(f\) :
\begin{align} \vvec{0} = \frac{df}{d\vvec{x}}(\vvec{x}^*) \label{eq:optimality-difffunc} \end{align}
was konkret heisst, dass der Nullvektor \(\vvec{0}\) gleich dem Gradient
\(\frac{df}{d\vvec{x}}\) ist an der optimalen Stelle \(\vvec{x}^*\).

Da wir aber bei unserem Problem \(\eqref{eq:convexproblem}\) diese
\textbf{unstetige}, \textbf{nicht-differenzierbare} Bestrafungsfunktion
\(\indf{C}\) eingebaut haben, ist dies nicht direkt mit der normalen
Differentiation zu machen. Man braucht in der konvexen Analysis eine
verallgemeinerte Ableitung - das \textbf{Subdifferential}, welches nicht
mehr nur einfache Steigungen (d.h. die Steigung für \(1\)-dimensionale
Funktionen \(f(x)\) oder allgemeiner der Gradient für \(n\)-dimensionale
Funktionen \(f(\vvec{x})\)) zurück geben kann sondern auch \textbf{ganze
Mengen} von solchen Steigungen. Das heisst, das Subdifferential an einem
Punkt ist eine Menge aller Gradienten an diesen Punkt der Funktion. Das
heisst direkt, dass eine Gleicheit zu \(\vvec{0}\) wie in
\(\eqref{eq:optimality-difffunc}\) nicht mehr richtig wäre und hier eine
Mengen-Inklusion \(\vvec{0} \in \dots\) stehen muss.

Anstatt \(\frac{d}{d\vvec{x}}\indf{C}\) nehmen wir einfach das
Subdifferential \(\partial_{\vvec{x}} \indf{C}\) und die Bedingung
\(\eqref{eq:optimality-difffunc}\) wird dann zu
\begin{align} \label{eq:optimality-strict-convex} \vvec{0} \in \frac{df}{d\vvec{x}}(\vvec{x}^*) + \partial_{\vvec{x}} \indf{C}(\vvec{x}^*) \end{align}

Nur was machen wir nun mit dieser \textbf{mengenwertigen Relation}?

Da man zeigen kann, dass das Subdifferential, also die mengenwertige
Ableitung, der Indikatorfunktion
\(\partial_{\vvec{x}} \indf{C}(\vvec{x})\) genau dem Normalkegel
\(\ncone{C}(\vvec{x})\) entspricht, können wir die obige Inklusion so
schreiben:
\begin{align} \vvec{0} \in \frac{df}{d\vvec{x}}(\vvec{x}^*) + \ncone{C}(\vvec{x}^*) \quad \Leftrightarrow \quad -\frac{df}{d\vvec{x}}(\vvec{x}^*) \in \ncone{C}(\vvec{x}^*) \label{eq:optimality-strict-convex-2} \end{align}

Das bringt uns nicht viel mehr ausser einer visuellen Erkenntnis durch
folgende Visualisierung:

\svgWithCaption{files/ConvexOptimizationProblem.svg}{Konvexes
Optimierungs Problem innerhalb der Menge \(\set{C}\) auf einer
\(2\)d-Funktion \(f(\vvec{x}) \in \mathbb{R}\). Der negative Gradient
liegt im Optimum \(\vvec{x}^*\) genau innerhalb des Normalkegels an
\(\vvec{x}^*\).}{}[fig:convex-opt-prob]

Mit der Beziehung zwischen \textbf{proximalem Punkt} und
\textbf{Normalkegel} \(\eqref{eq:prox-to-ncone}\) kriegen wir daraus
direkt eine \textbf{implizite Projektionsgleichung} für den optimalen
Punkt \(\vvec{x}^*\):
\begin{align} -\frac{df}{d\vvec{x}}(\vvec{x}^*) \in \ncone{C}(\vvec{x}^*) \quad \Leftrightarrow \quad \vvec{x}^* = \prox{C}(\vvec{x}^* - \frac{df}{d\vvec{x}}(\vvec{x}^*)), \end{align}
welche man iterative lösen kann, was zum
\textbf{Gradienten-Projektionsverfahren} führt (Gradient Projection
Algorithm).

\hypertarget{code-sample}{%
\section{Code Sample}\label{code-sample}}

Some inline code \texttt{int\ a;\ a\ +=\ "asd"}

\begin{Shaded}
\begin{Highlighting}[numbers=left,,]
\KeywordTok{class}\NormalTok{ LogicNode}
\OperatorTok{\{}
\KeywordTok{public}\OperatorTok{:}
\NormalTok{    EG\_DEFINE\_TYPES}\OperatorTok{();}

    \KeywordTok{using}\NormalTok{ InputSockets  }\OperatorTok{=} \BuiltInTok{std::}\NormalTok{vector}\OperatorTok{\textless{}}\NormalTok{LogicSocketInputBase}\OperatorTok{*\textgreater{};}
    \KeywordTok{using}\NormalTok{ OutputSockets }\OperatorTok{=} \BuiltInTok{std::}\NormalTok{vector}\OperatorTok{\textless{}}\NormalTok{LogicSocketOutputBase}\OperatorTok{*\textgreater{};}

\KeywordTok{public}\OperatorTok{:}
    \CommentTok{//! The basic constructor of a node.}
\NormalTok{    LogicNode}\OperatorTok{(}\NormalTok{NodeId id }\OperatorTok{=}\NormalTok{ nodeIdInvalid}\OperatorTok{)}
        \OperatorTok{:} \VariableTok{m\_id}\OperatorTok{(}\NormalTok{id}\OperatorTok{)}
    \OperatorTok{\{\}}

\NormalTok{    LogicNode}\OperatorTok{(}\AttributeTok{const}\NormalTok{ LogicNode}\OperatorTok{\&)} \OperatorTok{=} \ControlFlowTok{default}\OperatorTok{;}
\NormalTok{    LogicNode}\OperatorTok{(}\NormalTok{LogicNode}\OperatorTok{\&\&)}      \OperatorTok{=} \ControlFlowTok{default}\OperatorTok{;}

    \KeywordTok{virtual} \OperatorTok{\textasciitilde{}}\NormalTok{LogicNode}\OperatorTok{()} \OperatorTok{=} \ControlFlowTok{default}\OperatorTok{;}

    \CommentTok{//! The init function.}
    \KeywordTok{virtual} \DataTypeTok{void}\NormalTok{ init}\OperatorTok{()} \OperatorTok{=} \DecValTok{0}\OperatorTok{;}

    \CommentTok{//! The reset function.}
    \KeywordTok{virtual} \DataTypeTok{void}\NormalTok{ reset}\OperatorTok{()} \OperatorTok{=} \DecValTok{0}\OperatorTok{;}

    \CommentTok{//! The main compute function of this execution node.}
    \KeywordTok{virtual} \DataTypeTok{void}\NormalTok{ compute}\OperatorTok{()} \OperatorTok{=} \DecValTok{0}\OperatorTok{;}

\KeywordTok{public}\OperatorTok{:}
\NormalTok{    NodeId id}\OperatorTok{()} \AttributeTok{const} \OperatorTok{\{} \ControlFlowTok{return} \VariableTok{m\_id}\OperatorTok{;} \OperatorTok{\}}
    \DataTypeTok{void}\NormalTok{ setId}\OperatorTok{(}\NormalTok{NodeId id}\OperatorTok{)} \OperatorTok{\{} \VariableTok{m\_id} \OperatorTok{=}\NormalTok{ id}\OperatorTok{;} \OperatorTok{\}}

\KeywordTok{protected}\OperatorTok{:}
    \CommentTok{//! Registers input sockets for this node.}
    \KeywordTok{template}\OperatorTok{\textless{}}\KeywordTok{typename}\OperatorTok{...}\NormalTok{ Sockets}\OperatorTok{\textgreater{}}
    \DataTypeTok{void}\NormalTok{ registerInputs}\OperatorTok{(}\BuiltInTok{std::}\NormalTok{tuple}\OperatorTok{\textless{}}\NormalTok{Sockets}\OperatorTok{...\textgreater{}\&}\NormalTok{ sockets}\OperatorTok{)}
    \OperatorTok{\{}
        \VariableTok{m\_inputs} \OperatorTok{=}\NormalTok{ tupleUtil}\OperatorTok{::}\NormalTok{toPointers}\OperatorTok{\textless{}}
            \OperatorTok{[](}\KeywordTok{auto}\OperatorTok{*...}\NormalTok{ p}\OperatorTok{)} \OperatorTok{\{} \ControlFlowTok{return}\NormalTok{ InputSockets}\OperatorTok{\{}\NormalTok{p}\OperatorTok{...\};} \OperatorTok{\}\textgreater{}(}\NormalTok{sockets}\OperatorTok{);}
    \OperatorTok{\}}

    \CommentTok{//! Registers output sockets for this node.}
    \KeywordTok{template}\OperatorTok{\textless{}}\KeywordTok{typename}\OperatorTok{...}\NormalTok{ Sockets}\OperatorTok{\textgreater{}}
    \DataTypeTok{void}\NormalTok{ registerOutputs}\OperatorTok{(}\BuiltInTok{std::}\NormalTok{tuple}\OperatorTok{\textless{}}\NormalTok{Sockets}\OperatorTok{...\textgreater{}\&}\NormalTok{ sockets}\OperatorTok{)}
    \OperatorTok{\{}
        \VariableTok{m\_outputs} \OperatorTok{=}\NormalTok{ tupleUtil}\OperatorTok{::}\NormalTok{toPointers}\OperatorTok{\textless{}}
            \OperatorTok{[](}\KeywordTok{auto}\OperatorTok{*...}\NormalTok{ p}\OperatorTok{)} \OperatorTok{\{} \ControlFlowTok{return}\NormalTok{ OutputSockets}\OperatorTok{\{}\NormalTok{p}\OperatorTok{...\};} \OperatorTok{\}\textgreater{}(}\NormalTok{sockets}\OperatorTok{);}
    \OperatorTok{\}}

\KeywordTok{public}\OperatorTok{:}
    \CommentTok{//! Get the list of input sockets.}
    \AttributeTok{const}\NormalTok{ InputSockets}\OperatorTok{\&}\NormalTok{ getInputs}\OperatorTok{()} \AttributeTok{const} \OperatorTok{\{} \ControlFlowTok{return} \VariableTok{m\_inputs}\OperatorTok{;} \OperatorTok{\}}
\NormalTok{    InputSockets}\OperatorTok{\&}\NormalTok{ getInputs}\OperatorTok{()} \OperatorTok{\{} \ControlFlowTok{return} \VariableTok{m\_inputs}\OperatorTok{;} \OperatorTok{\}}

    \CommentTok{//! Get the list of output sockets.}
    \AttributeTok{const}\NormalTok{ OutputSockets}\OperatorTok{\&}\NormalTok{ getOutputs}\OperatorTok{()} \AttributeTok{const} \OperatorTok{\{} \ControlFlowTok{return} \VariableTok{m\_outputs}\OperatorTok{;} \OperatorTok{\}}
\NormalTok{    OutputSockets}\OperatorTok{\&}\NormalTok{ getOutputs}\OperatorTok{()} \OperatorTok{\{} \ControlFlowTok{return} \VariableTok{m\_outputs}\OperatorTok{;} \OperatorTok{\}}

    \CommentTok{//! Get the input socket at index \textasciigrave{}index\textasciigrave{}.}
    \AttributeTok{const}\NormalTok{ LogicSocketInputBase}\OperatorTok{*}\NormalTok{ input}\OperatorTok{(}\NormalTok{SocketIndex index}\OperatorTok{)} \AttributeTok{const}
    \OperatorTok{\{}
\NormalTok{        EG\_THROW\_IF}\OperatorTok{(}\NormalTok{index }\OperatorTok{\textgreater{}=} \VariableTok{m\_inputs}\OperatorTok{.}\NormalTok{size}\OperatorTok{(),} \StringTok{"Wrong index"}\OperatorTok{);}
        \ControlFlowTok{return} \VariableTok{m\_inputs}\OperatorTok{[}\NormalTok{index}\OperatorTok{];}
    \OperatorTok{\}}

    \CommentTok{//! Get the input socket at index \textasciigrave{}index\textasciigrave{}.}
\NormalTok{    LogicSocketInputBase}\OperatorTok{*}\NormalTok{ input}\OperatorTok{(}\NormalTok{SocketIndex index}\OperatorTok{)}
    \OperatorTok{\{}
\NormalTok{        EG\_THROW\_IF}\OperatorTok{(}\NormalTok{index }\OperatorTok{\textgreater{}=} \VariableTok{m\_inputs}\OperatorTok{.}\NormalTok{size}\OperatorTok{(),} \StringTok{"Wrong index"}\OperatorTok{);}
        \ControlFlowTok{return} \VariableTok{m\_inputs}\OperatorTok{[}\NormalTok{index}\OperatorTok{];}
    \OperatorTok{\}}

    \CommentTok{//! Get the output socket at index \textasciigrave{}index\textasciigrave{}.}
    \AttributeTok{const}\NormalTok{ LogicSocketOutputBase}\OperatorTok{*}\NormalTok{ output}\OperatorTok{(}\NormalTok{SocketIndex index}\OperatorTok{)} \AttributeTok{const}
    \OperatorTok{\{}
\NormalTok{        EG\_THROW\_IF}\OperatorTok{(}\NormalTok{index }\OperatorTok{\textgreater{}=} \VariableTok{m\_outputs}\OperatorTok{.}\NormalTok{size}\OperatorTok{(),} \StringTok{"Wrong index"}\OperatorTok{);}
        \ControlFlowTok{return} \VariableTok{m\_outputs}\OperatorTok{[}\NormalTok{index}\OperatorTok{];}
    \OperatorTok{\}}

    \CommentTok{//! Get the output socket at index \textasciigrave{}index\textasciigrave{}.}
\NormalTok{    LogicSocketOutputBase}\OperatorTok{*}\NormalTok{ output}\OperatorTok{(}\NormalTok{SocketIndex index}\OperatorTok{)}
    \OperatorTok{\{}
\NormalTok{        EG\_THROW\_IF}\OperatorTok{(}\NormalTok{index }\OperatorTok{\textgreater{}=} \VariableTok{m\_outputs}\OperatorTok{.}\NormalTok{size}\OperatorTok{(),} \StringTok{"Wrong index"}\OperatorTok{);}
        \ControlFlowTok{return} \VariableTok{m\_outputs}\OperatorTok{[}\NormalTok{index}\OperatorTok{];}
    \OperatorTok{\}}

\KeywordTok{protected}\OperatorTok{:}
\NormalTok{    NodeId }\VariableTok{m\_id}\OperatorTok{;}              \CommentTok{//!\textless{} The id of the node.}
\NormalTok{    InputSockets }\VariableTok{m\_inputs}\OperatorTok{;}    \CommentTok{//!\textless{} The input sockets.}
\NormalTok{    OutputSockets }\VariableTok{m\_outputs}\OperatorTok{;}  \CommentTok{//!\textless{} The output sockets.}
\OperatorTok{\};}
\end{Highlighting}
\end{Shaded}

\hypertarget{pdf-include-sample}{%
\section{PDF Include Sample}\label{pdf-include-sample}}

You can also include PDFs directly by: Selecting pages works only in
\texttt{latex} output.

\includePDF{files/PandocUsersGuide.pdf}[5]{10}[width=\textwidth]

\hypertarget{questionaire-sample}{%
\section{Questionaire Sample}\label{questionaire-sample}}

\hypertarget{personal}{%
\subsection*{Personal}\label{personal}}

\begin{itemize}
\tightlist
\item
  Name: \_\_\_\_\_\_\_\_\_\_\_\_\_\_\_\_\_\_
\item
  Email: \_\_\_\_\_\_\_\_\_\_\_\_\_\_\_\_\_
\end{itemize}

\hypertarget{how-hard-is-markdown}{%
\subsection*{How hard is Markdown?}\label{how-hard-is-markdown}}

\begin{itemize}
\tightlist
\item[$\square$]
  easy
\item[$\square$]
  medium hard
\item[$\square$]
  ridiculuous hard
\end{itemize}

\hypertarget{which-features-would-you-like-to-have-which-markdown-does-currently-not-support}{%
\subsection*{Which features would you like to have which Markdown does
currently not
support?}\label{which-features-would-you-like-to-have-which-markdown-does-currently-not-support}}

\rule{\textwidth}{0.5pt}
\rule{\textwidth}{0.5pt}
\rule{\textwidth}{0.5pt}
\rule{\textwidth}{0.5pt}
\rule{\textwidth}{0.5pt}

\hypertarget{tables}{%
\section{Tables}\label{tables}}

\hypertarget{html-table}{%
\subsection{HTML Table}\label{html-table}}

\begin{itemize}
\tightlist
\item
  Included html file as \texttt{html}.
\item
  Markdown citations/cross refeferences do not work inside.
\item
  HTML citations dont work.
\item
  Table caption is not parsed \texttt{table\_caption} not allowed as
  extension.
\end{itemize}

\begin{longtable}[]{@{}
  >{\raggedright\arraybackslash}p{(\columnwidth - 6\tabcolsep) * \real{0.1500}}
  >{\raggedright\arraybackslash}p{(\columnwidth - 6\tabcolsep) * \real{0.2000}}
  >{\raggedright\arraybackslash}p{(\columnwidth - 6\tabcolsep) * \real{0.3500}}
  >{\raggedright\arraybackslash}p{(\columnwidth - 6\tabcolsep) * \real{0.3000}}@{}}
\caption{Table by included \texttt{.html} file.}\tabularnewline
\toprule()
\begin{minipage}[b]{\linewidth}\raggedright
\textbf{Fruits asd}
\end{minipage} & \begin{minipage}[b]{\linewidth}\raggedright
\textbf{Vegetables}
\end{minipage} & \begin{minipage}[b]{\linewidth}\raggedright
\textbf{Constraints}
\end{minipage} & \begin{minipage}[b]{\linewidth}\raggedright
\textbf{Properties}
\end{minipage} \\
\midrule()
\endfirsthead
\toprule()
\begin{minipage}[b]{\linewidth}\raggedright
\textbf{Fruits asd}
\end{minipage} & \begin{minipage}[b]{\linewidth}\raggedright
\textbf{Vegetables}
\end{minipage} & \begin{minipage}[b]{\linewidth}\raggedright
\textbf{Constraints}
\end{minipage} & \begin{minipage}[b]{\linewidth}\raggedright
\textbf{Properties}
\end{minipage} \\
\midrule()
\endhead
Pre & Zuccini & ripe & long \\
First & \begin{minipage}[t]{\linewidth}\raggedright
\begin{enumerate}
\tightlist
\item
  Lorem ipsum.
\item
  Aliquam tinc.
\end{enumerate}
\end{minipage} & \begin{minipage}[t]{\linewidth}\raggedright
\begin{description}
\tightlist
\item[Definition 1]
\(\int_0^1{x^2} := \frac{1}{3}\)
\item[Definition 2]
\(x^2\) is a squared variable
\end{description}
\end{minipage} & \\
Second & \begin{minipage}[t]{\linewidth}\raggedright
\begin{itemize}
\tightlist
\item
  Lorem ipsum.
\item
  Mauris eu.
\end{itemize}
\end{minipage} & \begin{minipage}[t]{\linewidth}\raggedright
\begin{enumerate}
\tightlist
\item
  Lorem sit.
\item
  Dapibus.
\end{enumerate}
\end{minipage} & \begin{minipage}[t]{\linewidth}\raggedright
\begin{enumerate}
\tightlist
\item
  Aliquam risus.
\item
  Auctor neque.
\end{enumerate}
\end{minipage} \\
\bottomrule()
\end{longtable}

\hypertarget{table}{%
\subsection{\texorpdfstring{\LaTeX~Table}{~Table}}\label{table}}

\begin{itemize}
\tightlist
\item
  Included latex file as raw \texttt{latex}.
\item
  Converted from \texttt{.html} by \texttt{convert-tables.py} and
  \texttt{convert-tables.json}.
\item
  Latex citations do work inside.
\end{itemize}

\begin{longtable}[]{@{}
  >{\raggedright\arraybackslash}p{(\columnwidth - 6\tabcolsep) * \real{0.1500}}
  >{\raggedright\arraybackslash}p{(\columnwidth - 6\tabcolsep) * \real{0.2000}}
  >{\raggedright\arraybackslash}p{(\columnwidth - 6\tabcolsep) * \real{0.3500}}
  >{\raggedright\arraybackslash}p{(\columnwidth - 6\tabcolsep) * \real{0.3000}}@{}}
\caption{Table by included \texttt{.html} file.}\tabularnewline
\toprule()
\textbf{Fruits asd}
 & 
\textbf{Vegetables}
 & 
\textbf{Constraints}
 & 
\textbf{Properties}
 \\
\midrule()
\endhead
Pre & Zuccini & ripe & long \\ \midrule

First & 
\begin{enumerate}
\tightlist
\item
  Lorem ipsum.
\item
  Aliquam tinc.
\end{enumerate}
 & 
\begin{description}
\tightlist
\item[Definition 1]
\(\int_0^1{x^2} := \frac{1}{3}\)
\item[Definition 2]
\(x^2\) is a squared variable
\end{description}
 & \\ \midrule

Second & 
\begin{itemize}
\tightlist
\item
  Lorem ipsum.
\item
  Mauris eu.
\end{itemize}
 & 
\begin{enumerate}
\tightlist
\item
  Lorem sit.
\item
  Dapibus.
\end{enumerate}
 & 
\begin{enumerate}
\tightlist
\item
  Aliquam risus.
\item
  Auctor neque.
\end{enumerate}
 \\
\bottomrule()
\end{longtable}

\hypertarget{sec:multi-line-table}{%
\subsection{Markdown Table}\label{sec:multi-line-table}}

\begin{itemize}
\tightlist
\item
  Included markdown files.
\item
  Cross references do work here.
\end{itemize}

\begin{longtable}[]{@{}
  >{\centering\arraybackslash}p{(\columnwidth - 6\tabcolsep) * \real{0.1500}}
  >{\raggedright\arraybackslash}p{(\columnwidth - 6\tabcolsep) * \real{0.1000}}
  >{\raggedleft\arraybackslash}p{(\columnwidth - 6\tabcolsep) * \real{0.2750}}
  >{\raggedright\arraybackslash}p{(\columnwidth - 6\tabcolsep) * \real{0.3250}}@{}}
\caption{Here’s a multiline table without a header.}\tabularnewline
\toprule()
\endhead
First & row & 12.0 & Example of a row that spans multiple lines. \\
Second & row & 5.0 & Here’s another one. \\
\bottomrule()
\end{longtable}

\begin{longtable}[]{@{}
  >{\raggedright\arraybackslash}p{(\columnwidth - 4\tabcolsep) * \real{0.2000}}
  >{\raggedright\arraybackslash}p{(\columnwidth - 4\tabcolsep) * \real{0.1000}}
  >{\raggedright\arraybackslash}p{(\columnwidth - 4\tabcolsep) * \real{0.2625}}@{}}
\caption{Sample grid table.}\tabularnewline
\toprule()
\begin{minipage}[b]{\linewidth}\raggedright
Fruit
\end{minipage} & \begin{minipage}[b]{\linewidth}\raggedright
Price
\end{minipage} & \begin{minipage}[b]{\linewidth}\raggedright
Advantages
\end{minipage} \\
\midrule()
\endfirsthead
\toprule()
\begin{minipage}[b]{\linewidth}\raggedright
Fruit
\end{minipage} & \begin{minipage}[b]{\linewidth}\raggedright
Price
\end{minipage} & \begin{minipage}[b]{\linewidth}\raggedright
Advantages
\end{minipage} \\
\midrule()
\endhead
Bananas & \$1.34 & \begin{minipage}[t]{\linewidth}\raggedright
\begin{itemize}
\tightlist
\item
  built-in wrapper
\item
  bright color
\end{itemize}
\end{minipage} \\
Oranges & \$2.10 & \begin{minipage}[t]{\linewidth}\raggedright
\begin{itemize}
\tightlist
\item
  cures scurvy
\item
  tasty
\item
  Link \cref{sec:multi-line-table}
\end{itemize}
\end{minipage} \\
\bottomrule()
\end{longtable}

\hypertarget{references}{%
\chapter*{References}\label{references}}
\addcontentsline{toc}{chapter}{References}

\hypertarget{refs}{}
\begin{CSLReferences}{0}{0}
\leavevmode\vadjust pre{\hypertarget{ref-nuetzig_thesis_2016}{}}%
\CSLLeftMargin{{[}1{]} }%
\CSLRightInline{G. Nützi,
{‘\href{https://doi.org/10.3929/ethz-a-010662262}{Non-smooth granular
rigid body dynamics with applications to chute flows}’}, PhD thesis, ETH
Zurich; ETH Zürich, Zürich, 2016. }

\leavevmode\vadjust pre{\hypertarget{ref-rockafellar_convex_2015}{}}%
\CSLLeftMargin{{[}2{]} }%
\CSLRightInline{R. T. Rockafellar, \emph{Convex analysis}. Princeton:
Princeton University Press, 2015 {[}Online{]}. Available:
\url{https://www.degruyter.com/view/title/516543}}

\end{CSLReferences}

\end{document}